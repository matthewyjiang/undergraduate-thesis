\chapter{\leavevmode \newline Methods}
\label{chap:Methods}

\section{Safe Exploration on Granular Terrain}
\label{sec:Safe_Exploration_on_Granular_Terrain}

Navigation on granular terrain presents a sequential decision-making problem that can be framed as an optimization challenge with safety constraints. Effective navigation requires balancing exploration (discovering new information about the terrain) with exploitation (using known safe paths) while ensuring the robot never enters hazardous areas.

\subsection{Problem Formulation}
\label{sec:Problem_Formulation}

When navigating on granular terrain such as sandy or icy surfaces, the robot must make estimations on key metrics such as terrain shear strength and ice content. These properties can vary significantly across the terrain, creating regions of varying risk. The navigation task can be formalized as finding an optimal path through a two-dimensional descrete grid where:

\begin{enumerate}
    \item Each point has an associated risk/safety level that is initially unknown
    \item Safety constraints must be satisfied throughout navigation
    \item Information about terrain properties must be collected during movement
\end{enumerate}

This fits the safety-constrained optimization framework described by Sui et al. \cite{safeopt}, referred to as "SafeOPT", where the goal is to estimate an unknown function, where the metric represented by the function has a direct correlation with the risk associated at each point in the function domain. The SafeOPT algorithm greedily selects points of interest, balancing exploration and exploitation, while guaranteeing that these points fall within a "safe set", a set of points that satisfy the safety constraints of navigation. 
\subsection{SafeOPT Algorithm for Terrain Navigation}
Modeling the terrain properties as a Gaussian process (GP) allows for:
\begin{enumerate}
    \item Representing uncertainty about terrain properties across the entire environment
    \item Making confidence-based predictions about unvisited areas
    \item Incrementally expanding a set of locations certified as safe
\end{enumerate}
SafeOPT maintains three subsets of the discrete map domain $D$ at each timestep $t$: 
\begin{itemize}
    \item \textbf{Safe Set}: $S_t = \{ x \in D \mid x < \epsilon \}$, where $\epsilon$ is a threshold value that defines the maximum allowable risk
    \item \textbf{Expander Set} $G_t = \{ x \in S_t \mid g_t(x) > 0 \}$: Where $g_t(x)$ quantifies the potential enlargement of the current safe set after sampling a new decision $x$ \cite{safeopt}, or points that could potentially expand the safe set when observed.
    \item \textbf{Maximizer Set} $M_t = \{ x \in S_T \mid \text{where $f$ is} \}$: \cite{safeopt}
\end{itemize}
To make progress, the algorithm iteratively selects the location with maximum uncertainty among these sets, balancing between:
\begin{itemize}
    \item Reducing uncertainty about potential maximizers (exploitation)
    \item Expanding the set of known safe regions (exploration)
\end{itemize}
In the two-dimensional granular terrain navigation problem, we define the unsafe set using the sample set $D$ and safe set $S_t$:
\[ U_t = D \setminus S_t \]

$U_t$ will be used later to define the geometry of unsafe regions on the map.




