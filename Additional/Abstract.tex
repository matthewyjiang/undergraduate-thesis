% \chapter*{Abstract}
\chapter*{Abstract}
\chaptermark{Abstract}
\label{chap:Abstract}
\addcontentsline{toc}{chapter}{Abstract}
%Insert your abstract below the line
%-------------------------------------------

This thesis presents a novel approach to autonomous navigation for planetary exploration that addresses the critical limitations of traditional wheeled rovers. Building upon NASA's LASSIE (Legged Autonomous Surface Science In Analogue Environments) project, we develop an integrated algorithm for legged rovers that utilizes proprioceptive leg motor data to assess geotechnical properties of planetary regolith while simultaneously planning safe and information-rich exploration paths.

The algorithm operates within the LASSIE exploration loop, combining measurements from compression and shearing tests performed by the robot's legs with conventional obstacle detection systems (cameras, lidar, radar). This dual-objective approach not only plans paths that avoid dangerous terrain but also strategically selects future testing locations to maximize knowledge acquisition about the environment.

Extensive simulation and testing in analog environments demonstrates how this algorithm enhances autonomous capabilities for legged rovers navigating unknown granular terrain. Results show significant improvements in both terrain assessment accuracy and exploration efficiency compared to traditional methods that rely solely on visual data.

By enabling rovers to actively probe their environment and make informed decisions based on physical terrain properties, this research addresses a fundamental challenge in planetary exploration and lays groundwork for the identification of safe locations for future human infrastructure. The algorithmic framework presented provides an essential capability for sustainable robotic and eventual human presence beyond Earth.
