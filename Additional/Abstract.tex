\chapter*{Abstract}
\chaptermark{Abstract}
\label{chap:Abstract}
\addcontentsline{toc}{chapter}{Abstract}

This thesis presents \algonamefull{} (\algoname{}), a navigation framework for autonomous exploration in granular environments where proprioceptive sensing is the main mode of terrain mapping. The method is motivated by challenges faced in planetary exploration, particularly in scenarios where deformable terrain properties cannot be reliably inferred from visual input. Developed within the context of NASA’s LASSIE (Legged Autonomous Surface Science In Analogue Environments) project, \algoname{} enables a legged robot to traverse unfamiliar regolith terrain using only proprioceptive feedback from its limbs.

The core contribution is a dual-layer architecture combining global terrain expansion with real-time reactive control. The global layer models terrain risk using a Gaussian Process updated through in-place mechanical tests at each step. It incrementally expands a certified safe set of traversable regions using a confidence-aware exploration strategy inspired by Safe Bayesian Optimization. The local control layer uses a diffeomorphism-based reactive controller adapted from Voronoi and power diagram methods, allowing the robot to navigate through complex, concave environments without requiring global replanning.

Simulations demonstrate that \algoname{} successfully balances exploration and exploitation, avoids unsafe regions, and generalizes across navigation and pure exploration tasks. The algorithm achieves real-time performance and robustness without vision, supporting future missions in low-light or dust-obscured extraterrestrial environments. This work establishes a foundation for proprioception-driven planning, expanding autonomous mobility capabilities in safety-critical planetary operations.
